\documentclass{article}
\usepackage[english]{babel}
\usepackage{amsthm}
\usepackage[margin=1in]{geometry}
\usepackage{setspace}
\usepackage{fancyhdr}
\pagestyle{fancy}
\fancyhf{}
\rfoot{\thepage}
\setstretch{1.0}
\theoremstyle{definition}
\newtheorem{definition}{Definition}[section]
\newtheorem{theorem}{Theorem}
\theoremstyle{theorem}
\theoremstyle{remark}
\newtheorem*{remark}{Remark}

\begin{document}
\begin{definition}[Context Free Grammar]
    A Context Free Grammar (CFG) $G$ is a 4-tuple $(V, \Sigma, R, S)$ where:
    \begin{enumerate}
        \item  $V$ finite set of variables
        \item  $\Sigma$ finite set of terminal symbols
        \item  $R$ finite set of RULES ($V \rightarrow (V \cup \Sigma)^{*}$)
        \item  $S$ the start variable
    \end{enumerate}
\end{definition}


\begin{definition}[Pumping Lemma for CFLs]
    For every CFL $A$, there is a $p$ such that if $s \in A$ and $|s| \geq p$ then $s = uvxyz$ where
    \begin{enumerate}
        \item $uv^ixy^iz \in A$ for all $i \geq 0$
        \item $vy \neq \sigma$
        \item $|vxy| \leq p$
    \end{enumerate}

\begin{definition}[Turing Machine]
A Turing machine is a three-tuple, $(\Sigma, \Lambda)$
\end{definition}

\end{definition}
\end{document}