\documentclass{article}
\usepackage[english]{babel}
\usepackage{amsthm}
\usepackage{amsmath}
\usepackage[margin=1in]{geometry}
\usepackage{setspace}
\usepackage{fancyhdr}
\pagestyle{fancy}
\fancyhf{}
\rfoot{\thepage}
\setstretch{1.0}
\theoremstyle{definition}
\newtheorem{definition}{Definition}[section]
\theoremstyle{remark}
\newtheorem*{remark}{Remark}

\begin{document}
\section*{Intro}

This serves as a personal note for learning about Temporal Logic of Action (TLA).

\section*{Definitions that will come handy}
I rather lay out all the definitions needed for the learning of the subject. It always feels good to have a place where all of the newly-learned definitions live, so I can come back and review each of them anytime I want.

I will be just laying out definitions and sometimes with a few examples. This section should be treated as a reference (usually at the end of a book).

Assuming we all have some understanding on \underline{propositional logic}, such as familiarity with concepts like \underline{premises}, \underline{conclusion} and \underline{deduction}, we can start with a few basic definitions.

\begin{definition}[Predicate]

A predicate is a function from variables to $\left\{TRUE, FALSE\right\}$

For example, $P(X) := X \text{is the multiple of two}$ is a predicate depending on the variable $X$. 

Depending on the input, the predicate can be true or false. In this example, $P(1)$ is false and $P(2)$ is false.


\end{definition}
\end{document}